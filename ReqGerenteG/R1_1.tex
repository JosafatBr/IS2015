

\section{Requerimentos Funcionales del gerente general}
\begin{problemas}[BlueViolet]{}
\problema{RGG1}{Podrá gestionar los usuarios de todas las sucursales.}{RG1}{Gerente general}{A}

\problema{RGG2}{Podrá gestionar el catálogo de panes.}{}{Gerente general}{MA}

\problema{RGG3}{Podrá monitorear el reporte de ventas por sucursal.}{}{Gerente general}{M}

\problema{RGG4}{Podrá descargar el reporte de ventas por sucursal.}{}{Gerente general}{A}

\problema{RGG5}{Podrá visualizar gráficas parciales de todos los reportes.}{}{Gerente general}{B}

\problema{RGG6}{Podrá modificar cantidad exacta de pan a producir.}{}{Gerente general}{B}

\problema{RGG7}{Podrá modificar cantidad a mandar a cada panadería.}{}{Gerente general}{B}

\problema{RGG8}{Podrá efectuar cualquier actividad de empleado o gerente de sucursal}{}{Gerente general}{A}

\problema{RGG9}{Podrá ver los ingresos de venta de pan por tipo o totales.}{}{Gerente general}{A}

\problema{RGG10}{Podrá ver los reportes parciales por día,mes año}{}{Gerente general}{A}

\problema{RGG11}{Podrá ver esquemas deproducción y demanda.}{}{Gerente general}{M}

\end{problemas}

% Definir un problema (se usa con el environment problemas
% Usar: \problema{id o Nombre}{Descripción}{Origen}{Prioridad}
%
% Ejemplo: MB - Muy Baja, B - Baja, M - Media, A - Alta, MA - Muy Alta.