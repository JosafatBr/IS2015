\section{Requerimentos del gerente de sucursal}
\begin{problemas}[BlueViolet]{}
\problema{Dar de alta empleado}{El gerente dará de alta al empleado ingresando su nombre, edad y puesto.}{Via formulario}{M}

\problema{Dar de baja a empleado.}{El gerente podrá dar de baja un empleado por medio de su ID.}{Via formulario}{M}

\problema{Visualizar datos del empelado}{El gerente visualizará los datos del empleado como su nombre, edad, puesto, salario y su fotografía.}{Una pantalla}{M}

\problema{Modificar datos de empleado.}{El gerente podrá modifica los datos del empleado como su nombre, edad o puesto.}{Via formulario}{M}

\problema{Agregar pan al catálogo}{El gerente podrá registrar un pan insertando su nombre, precio y una imagen de este.}{Via formulario}{A}

\problema{Modificar pan al catálogo}{El gerente podrá modificar los panes registrados en el catálogo como su fotografía, precio o nombre.
}{Vía formulario}{A}

\problema{Eliminar pan del catálogo}{El gerente podrá eliminar pan registrado en el catálogo por medio del ID.}{Via formulario}{A}

\problema{Consultar ventas}{EL gerente podrá consultar las ventas realizadas en la sucursal en cualquier momento del día.}{Una Pantalla}{A}

\problema{Dar de baja venta.}{El gerente podrá cancelar ventas por medio del identificador de esta.}{Vía formulario}{M}

\problema{Consultar contabilidad}{El gerente podrá consulta la contabilidad de su sucursal.}{Una pantalla}{M}


\problema{Consultar esquema de producción}{El gerente podrá consultar el esquema de producción de la panadería para visualizar que pan se produjo.}{Una pantalla}{A}

\problema{Consultar esquema de demanda}{El gerente podrá consultar el esquema de demanda.}{Una pantalla}{A}
\end{problemas}


% Definir un problema (se usa con el environment problemas
% Usar: \problema{id o Nombre}{Descripción}{Origen}{Prioridad}
%
% Ejemplo: MB - Muy Baja, B - Baja, M - Media, A - Alta, MA - Muy Alta.